\documentclass{article}
 
\title{Attitude control of miniature quadrotor}
\author{Nestor A Eduartes}
 
\begin{document}
 
\maketitle
 
\begin{abstract}
Two control schemes are evaluated for the attitude control of a hovering quad rotor based on the non linear lagrangian dynamic model. The system model is developed based on a set of parameter for a theoretical quad rotor. Attitude control is performed using a set of independent pole placement state feedback controlled based on a linearized decoupled model simplification. A second controller is designed using an optimal linear quadratic state feedback regulator, based on a linear model. The performance of both controllers are assessed by the response to small disturbances around the hovering state.
\end{abstract}

\section{Introduction}
The quad rotor is a type of rotorcraft where the lift is supplied by four rotors. These four rotors are arranged in two pairs of opposite rotation propellers. Two of them rotate in a counter]clockwise manner, and the other two in a clockwise manner.

The total lift generated by the four propellers overcomes the gravity pull on the structure, while the ratio between the lift of opposing propellers define the net torque on the main orientation axis. These net torque generated by the rotors define a roll, pitch or yaw rotation together with a linear displacement. The quad rotor has an under‐actuated, unstable dynamics that requires the use of a controller for stable operation.

\section{Dynamic Model}
A simplified model of the quad rotor is provided in the reference paper. The dynamic model is derived by the inertia of the body and the four rotors, as well as the lift and drag generated by the propellers. State variables of the quad rotor consists of the position and attitude of the structure, as well as their rate of change. The complete state variable description comprises 12 states, not including any additional states required for the description of the rotor dynamics.

\section{Linearized Controller}

\section{Pole Placement Controller Design}

\section{Linear Quadratic Regulator}

 
 
\end{document}